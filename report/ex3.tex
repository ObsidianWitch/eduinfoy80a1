\section{Exercise 3: Coupling \& Cohesion}

\subsection{Levels of coupling and cohesion}

\textbf{Cohesion} refers to how related attributes and methods are
within a class and if they are responding to the class' intention.
\textbf{High cohesion}, which is preferable, means that a class does one
specific job well. Low cohesion, which should be avoided, means that the
class is not focused on one thing and could be refactored. For example,
an \emph{User} class containing attributes such as the name or the email
address and methods for sending messages to other users has a low
cohesion, since message handling could be refactored and done by other
classes.\\

\textbf{Coupling} refers to the dependencies between classes. Highly
coupled classes cannot be used independently, consequently, changes to
those classes are difficult to make without having to modify all the
dependent classes. It's also hard to reuse and test classes with high
coupling because all the dependencies must be carried with them. So, we
should try to have \textbf{low coupling} between the modules in our
programms.

\subsection{Cohesion in jEdit}

\subsubsection{MiscUtilities}

\subsubsection{GUIUtilities}

\subsubsection{VFSFile}

\subsection{Coupling in jEdit}

\newpage
