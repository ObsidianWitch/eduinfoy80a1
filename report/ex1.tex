\section{Exercise 1: Find instances of Design Patterns}

In this exercise, we had to find instances of design patterns in jEdit's
code, jEdit being an open source text editor written in Java.

\subsection{Singleton}
The singleton pattern is categorized as a creational pattern because it
ensures that a specific class is \textbf{instantiated} only once.\\

It is often criticized because an instance of a class using this pattern
can be used like a global variable. A global variable causes problem for
unit testing, because it can be modified by everyone and causes
unpredictability in the state of the system. It also introduces coupling
with classes using it, which renders them hardly reusable.\\ % TODO ref

While using global variables may be seen as bad practice, and while
there are alternatives (e.g. dependency injection), these alternatives
may bring drawbacks such as having the code difficult to read, or the
intent of the developpers may not be easy to grasp at first glance.
Global variables may even be the simplest way of using tools such as a
Logging classes which is not directly part of the application but only
here to help debugging. \cite{cite:plSingletonGlobal} \cite{cite:soSingleton}
\cite{cite:seGlobal}\\

The example we chose for a singleton implementation is the
\emph{PluginManager}\footnotemark[1]{} class.
This class handles the window (extends \emph{JFrame}) where plugins are
installed and updated. The singleton pattern is used in this
case so that only one plugin manager window can be instantiated and
displayed at once when invoking it via the menu (\emph{Plugins -> Plugin
Manager...}). The following methods are involved :

\footnotetext[1]{\emph{org.gjt.sp.jedit.pluginmgr} package}

\begin{itemize}\itemsep1pt
    \item the \emph{showPluginManager()} method instantiate a \emph{PluginManager}
    if it has not already been instantiated, else it brings the plugin manager
    window to the front;

    \item The \emph{getInstance()} method retrieve the current instance of
    \emph{PluginManager} (can be null if it has not yet been instantiated or if
    the instance has been disposed of with the \emph{dispose()} method).
\end{itemize}

\begin{figure}[h!]
    \includegraphics[width=0.4\textwidth]{images/singleton.png}
    \centering
    \caption{Singleton example class diagram}
\end{figure}

\begin{framehint}
    The following classes are also implementing a singleton pattern, but they
    will not be described in details.

    \begin{itemize}\itemsep1pt
        \item \emph{ReflectManager} (\emph{org.gjt.sp.jedit.bsh})
        \item \emph{KillRing} (\emph{org.gjt.sp.jedit.buffer})
        \item \emph{ModeProvider} (\emph{org.gjt.sp.jedit.syntax})
        \item \emph{TransferHandler} (\emph{org.gjt.sp.jedit.datatransfer})
        \item \emph{DockableWindowFactory} (\emph{org.gjt.sp.jedit.gui})
    \end{itemize}
\end{framehint}
\newpage

\subsection{Abstract Factory}
\noindent The abstract factory pattern is a creational pattern which helps creating
related objects.\\

The example found in jEdit's code is centered around the
\emph{StatusWidgetFactory} interface which must be implemented by all the
factories related to constructing statusbar widgets. Examples of factories
implementing it are \emph{ErrorsWidgetFactory}, \emph{ClockWidgetFactory},
\emph{LastModifiedWidgetFactory}. Its purpose is to instantiate StatusBar
widgets without having to specify the concrete class needed to construct it.

\begin{figure}[h!]
    \includegraphics[width=0.8\textwidth]{images/abstractfactory.png}
    \centering
    \caption{Abstract Factory example class diagram}
\end{figure}

\begin{framehint2}
    \textbf{N.B.} Statusbar widgets also seem to be used as services and are
    listed in the \emph{services.xml} file. This file specifies the services'
    classes or factory classes, names and ways of instantiating them
    (constructor to call).
    Each service is a singleton handled by the \emph{ServiceManager}.
    By loading services through an XML file, the program handles the
    instantiation of services the same way for all services. Furthermore, the
    program's code does not need to be modified to add a service, only the
    xml must be modified; hence, the \emph{ServiceManager} does not need to
    depend on all the services' factories and the coupling is reduced.
\end{framehint2}
\newpage

\subsection{Observer}
The Observer is a behavioral pattern, it is justified by the fact that
subjects communicate with observers to which they register.\\

The example we chose for this pattern is composed of the
\emph{Autosave}\footnotemark[1]{}, \emph{ActionListener} \& \emph{Timer} classes.
The \emph{Autosave} class' purpose is to automatically save all buffers with
unsaved changes after a certain amount of time has elapsed (default
value is 30 secondes). This class implements the \emph{ActionListener}
(awt) interface which contains the \emph{actionPerformed()} method. This
method is called by a subject, here the subject being a \emph{Timer}
(swing) which calls \emph{actionPerformed()} after the specified amount
of time has elapsed. Finally, \emph{actionPerformed()} calls the
\emph{autosave()} method on all the buffers (the buffers are globally
accessible from \emph{jEdit}'s \emph{getBuffer()} static method). The
Observer pattern is a way of avoiding busy-waiting: instead of checking
repeatedly whether the time has elapsed, it lets the \emph{Timer} notify
that he has finished counting to its listeners (\emph{Autosave}).

\footnotetext[1]{\emph{org.gjt.sp.jedit.Autosave}}

\begin{figure}[h!]
    \includegraphics[width=0.8\textwidth]{images/observer.png}
    \centering
    \caption{Observer example class diagram}
\end{figure}

\begin{framehint}
    The following classes are also implementing an observer pattern but were
    not chosen as our example.

    \begin{itemize}\itemsep1pt
        \item \emph{RegistersListener} as the observer interface,
        \emph{JEditRegistersListener} as the concrete observer, and
        \emph{Registers} as the subject

        \item \emph{HelpHistoryModelListener} as the observer interface,
          \emph{HelpViewer} as the concrete observer and \emph{HelpHistoryModel}
          as the subhect (\emph{org.gjt.sp.jedit.help})
    \end{itemize}
\end{framehint}
\newpage

\subsection[Adapter]{Adapter \cite{cite:soAdapterDiff}}
The adapter pattern is a structural pattern used to convert an interface into
another interface which a client expects.\\

One example of an adapter used in jEdit's code is \emph{EditAction}'s
inner class \emph{Wrapper}. It adapts an \emph{ActionContext} object to
the \emph{ActionListener} interface (awt). The purpose of this adapter
is to perform an \emph{EditAction} with the help of the
\emph{ActionContext} instance when an action event is received as part
of awt's \emph{ActionListener}.

\begin{figure}[h!]
    \includegraphics[width=0.7\textwidth]{images/adapter.png}
    \centering
    \caption{Adapter example class diagram}
\end{figure}

\begin{framehint}
    Other examples of the adapter pattern where found in jEdit's code :

    \begin{itemize}\itemsep1pt
        \item \emph{TextArea.CharIterator} which is the adapter class which
        adapts the \emph{CharSequence} (\emph{java.lang}) interface to the
        \emph{CharacterIterator} (\emph{java.text}) interface;

        \item \emph{PositionManager.PosTopHalf} which adapts
        \emph{PositionManager.PosBottomHalf}'s interface to the \emph{Position}
        interface.
    \end{itemize}
\end{framehint}

\begin{framehint2}
    \textbf{N.B.} jEdit possesses classes with ``Adapter'' in their class name.
    They currently have the same meaning as the Adapters used in \emph{awt}
    (e.g. \emph{ComponentAdapter}, \emph{WindowAdapter}), and are not
    implementations of the Adapter design pattern.
    These adapters are used as a convenience for creating listener objects, and
    their methods are empty \cite{cite:javaDocWindowAdapter}.
    Some examples of such classes in jEdit are \emph{BufferAdapter},
    \emph{BufferSetAdapter}, \emph{JEditVisitorAdapter}.
    They also have another purpose: as stated in \emph{bufferListener}'s
    documentation, this interface may change in the future and thus
    \emph{BufferAdapter} should be subclassed instead.
    By using \emph{BufferAdapter} instead of \emph{BufferListener}, developpers
    will only have to modify \emph{BufferAdapter} if \emph{BufferListener} is
    modified.
\end{framehint2}
\newpage

\subsection{Visitor}
The visitor design pattern is a behavioral pattern, its purpose being to
separate algorithms from an object structure (e.g. composite).\\

In jEdit's case, the object structure is composed of the following classes:
\emph{View}, \emph{EditPane} and \emph{TextArea}.
The visitor interface is called
\emph{JEditVisitor}\footnote{\emph{org.gjt.sp.jedit.visitors} package} and some
concrete implementations are \emph{SaveCaretInfoVisitor} and some anonymous
classes. The purpose of this design pattern here is to have multiple operations
available for an element without having to modify it. As of jEdit's 5.2 version :

\begin{itemize}\itemsep1pt
    \item  \emph{View} is only visited to visit its \emph{EditPanes} and the
    \emph{EditPanes}' \emph{TextAreas}
    \item \emph{TextArea} does not seem to have any behaviour added through the visitors for the moment
    \item \emph{EditPanes} as several behaviours added through
    \emph{SaveCaretInfoVisitor} and other anonymous classes implementing
    \emph{JEditVisitor} (e.g. in \emph{BufferSetManager}, \emph{CompleteWorld}
    and \emph{Buffer}).
\end{itemize}

\begin{figure}[h!]
    \includegraphics[width=0.8\textwidth]{images/visitor.png}
    \centering
    \caption{Visitor example class diagram}
\end{figure}

As for the differences with the original pattern, the elements from the object structure being visited do not implement a common interface.
\newpage
